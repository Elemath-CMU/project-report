\maketitle
\makesignature

\ifproject
\begin{abstractTH}
ในปัจจุบัน การเรียนรู้เรื่อง “เศษส่วน” สำหรับนักเรียนระดับประถมศึกษายังคงเป็นเรื่องที่เข้าใจได้ยาก เนื่องจากเป็นแนวคิดเชิงนามธรรมที่นักเรียนไม่คุ้นเคย และสื่อการสอนที่มีอยู่ส่วนใหญ่มักมุ่งเน้นการผลลัพธ์ เช่น การหาคำตอบที่ถูกต้องหรือการทำโจทย์ให้ได้คะแนน มากกว่าการสร้างความเข้าใจในหลักการและกระบวนการคิด ทำให้นักเรียนไม่สามารถเชื่อมโยงเศษส่วนกับสถานการณ์ในชีวิตจริงได้

ด้วยเหตุนี้ ผู้จัดทำจึงเสนอแนวคิดในการพัฒนาสื่อการสอนแบบเปิด เพื่อให้นักเรียนได้ลงมือทดลอง คิดแก้ปัญหา และค้นพบความรู้ด้วยตนเองผ่านการจำลองสถานการณ์ต่าง ๆ ที่เกี่ยวข้องกับเศษส่วน โดยเน้นให้เด็กได้สังเกตและเข้าใจแนวคิดหลักโดยไม่ถูกจำกัดด้วยวิธีการคงที่หรือตัวเลือกที่ตายตัว ซึ่งจะเป็นประโยชน์ต่อทั้งนักเรียน และครูที่จะได้เครื่องมือช่วยสอนที่มีประสิทธิภาพ สอดคล้องกับหลักสูตร

โดยแนวทางที่เลือกใช้คือการออกแบบและพัฒนาสื่อการสอนเชิงโต้ตอบที่เน้นการสะท้อนสถานการณ์จริง เช่น การวัดพื้นที่ การแบ่งสิ่งของ ในส่วนของทางเลือกอื่นที่พิจารณาแทนได้คือ การใช้สื่อเทคโนโลยี เช่น แอปพลิเคชันหรือเกม รวมทั้งการใช้สื่อวิดีโอและภาพเคลื่อนไหว

เงื่อนไขและข้อจำกัดที่มีได้แก่ ความครอบคลุมของเนื้อหาที่ต้องจำเพาะเจาะจงไปที่เนื้อหาเรื่องเศษส่วนเพียงเรื่องเดียว เพราะเป็นเรื่องที่เด็กต้องใช้เวลาในการเชื่อมโยงเพื่อเข้าใจหลักการของเศษส่วนอย่างแท้จริง นอกจากนี้ สื่อการสอนนี้อาจเป็นสื่อที่แปลกใหม่สำหรับครูผู้สอน จึงจำเป็นต้องมีการแนะแนวหรือจัดทำคู่มือเพื่อให้การใช้งานมีประสิทธิภาพ
\end{abstractTH}

\begin{abstract}
Currently, learning about fractions for elementary students remains challenging, as it is an abstract concept that students are unfamiliar with. Most existing teaching materials tend to focus on outcomes such as finding the correct answers or solving problems to earn scores rather than fostering understanding of the principles and thought processes. This results in students being unable to connect fractions with real-life situations.

For this reason, the author proposes the idea of developing open-ended teaching materials, allowing students to experiment, think critically, solve problems, and discover knowledge by themselves through simulations of various situations related to fractions. The focus is on encouraging children to observe and grasp the core concepts without being constrained by fixed methods or predetermined choices. This approach would benefit both students by supporting deeper understanding and teachers by providing an effective teaching tool aligned with the curriculum.

The chosen approach is to design and develop interactive teaching materials that reflect real-world situations, such as measuring areas or dividing objects. Alternative options considered include using technology-based media such as applications or games, as well as video and animation media.

The conditions and limitations include the scope of the content, which must specifically focus only on fractions, since students need time to build connections and gain a true understanding of fraction principles. Moreover, because this teaching material may be new to teachers, it is necessary to provide guidance or manuals to ensure effective use.
\end{abstract}

\iffalse
\begin{dedication}
This document is dedicated to all Chiang Mai University students.

Dedication page is optional.
\end{dedication}
\fi % \iffalse

\begin{acknowledgments}
Your acknowledgments go here. Make sure it sits inside the
\texttt{acknowledgment} environment.

\acksign{2020}{5}{25}
\end{acknowledgments}%
\fi % \ifproject

\contentspage

\ifproject
\figurelistpage

\tablelistpage
\fi % \ifproject

% \abbrlist % this page is optional

% \symlist % this page is optional

% \preface % this section is optional
