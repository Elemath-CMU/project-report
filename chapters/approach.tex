\chapter{\ifproject%
\ifenglish Project Structure and Methodology\else โครงสร้างและขั้นตอนการทำงาน\fi
\else%
\ifenglish Project Structure\else โครงสร้างของโครงงาน\fi
\fi
}

ในบทนี้จะกล่าวถึงขั้นตอนการดำเนินงานในโครงงานนี้

\makeatletter

% \renewcommand\section{\@startsection {section}{1}{\z@}%
%                                    {13.5ex \@plus -1ex \@minus -.2ex}%
%                                    {2.3ex \@plus.2ex}%
%                                    {\normalfont\large\bfseries}}

\makeatother
%\vspace{2ex}
% \titleformat{\section}{\normalfont\bfseries}{\thesection}{1em}{}
% \titlespacing*{\section}{0pt}{10ex}{0pt}

\section{การค้นคว้าข้อมูล}

\begin{figure}
\begin{center}
\includegraphics{800px-Briny_Beach.jpg}
\end{center}
\caption[Poem]{The Walrus and the Carpenter}
\label{fig:walrus}
\end{figure}

\subsection{วิเคราะห์ปัญหา}
 โครงงานนี้เริ่มต้นจากความสงสัยว่า การเรียนรู้วิชาคณิตศาสตร์เรื่องเศษส่วนสำหรับเด็กนักเรียนชั้นประถมศึกษาในปัจจุบันนั้นมีปัญหาอย่างไรบ้าง
 ...คิดไม่ออก
\subsection{วิเคราะห์วิธีแก้ปัญหาที่มีอยู่ในปัจจุบัน}
 
\subsection{สรุปปัญหาเบื้องต้น}

\section{การลงพื้นที่สำรวจ}

\subsection{เลือกพี้นที่สำรวจ}

\subsection{วางแผนออกสำรวจ}


