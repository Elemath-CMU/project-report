\chapter{\ifproject%
\ifenglish Project Structure and Methodology\else โครงสร้างและขั้นตอนการทำงาน\fi
\else%
\ifenglish Project Structure\else โครงสร้างของโครงงาน\fi
\fi
}

ในบทนี้จะกล่าวถึงขั้นตอนการดำเนินงานในโครงงานนี้

\makeatletter

% \renewcommand\section{\@startsection {section}{1}{\z@}%
%                                    {13.5ex \@plus -1ex \@minus -.2ex}%
%                                    {2.3ex \@plus.2ex}%
%                                    {\normalfont\large\bfseries}}

\makeatother
%\vspace{2ex}
% \titleformat{\section}{\normalfont\bfseries}{\thesection}{1em}{}
% \titlespacing*{\section}{0pt}{10ex}{0pt}

\section{การค้นคว้าข้อมูล}
ในช่วงเริ่มต้นของโครงงานนี้ พวกเรามุ่งเน้นการค้นคว้าข้อมูลที่เกี่ยวข้องกับโครงงานนี้ 
เพื่อให้เรามีข้อมูลมากเพียงพอที่จะเข้าใจถึงปัญหาของการรู้วิชาคณิตศาสตร์เรื่องเศษส่วนและแนวทางการแก้ไขปัญหาที่มีอยู่ในปัจจุบัน

\subsection{วิเคราะห์ปัญหา}
 พวกเราตั้งข้อสงสัยว่า การเรียนรู้วิชาคณิตศาสตร์เรื่องเศษส่วนสำหรับเด็กนักเรียนชั้นประถมศึกษาในปัจจุบันนั้นมีปัญหาอย่างไรบ้าง
 หลังจากนั้นเราได้ติดต่อพูดคุยกับนักวิชาการของ สสวท. เพื่อสอบถามเกี่ยวกับข้อสงสัยนี้และได้ข้อมูลว่า การเรียนรู้วิชาคณิตศาสตร์เรื่องเศษส่วนนั้นเป็นปัญหาที่พบในเด็กนักเรียนชั้นประถมศึกษาทั่วโลก
 เนื่องจากเนื้อหาในเรื่องนี้ยากต่อการทำให้เด็กเข้าใจและเห็นภาพที่ถูกต้องของเศษส่วน เพราะเศษส่วนมีรูปแบบจำนวนที่ต่างจากคณิตศาสตร์ในเรื่องก่อนหน้าที่เป็นจำนวนเต็ม และอาจมีหลักการที่ดูขัดแย้งกับสิ่งที่เขาเคยเรียนมา 
 จึงยากที่จะทำให้เด็กทุกคนเข้าใจพร้อมๆกันและไม่สามารถทำเด็กทุกคนเข้าใจเรื่องนี้ด้วยวิธีสอนเดียวกันได้

\subsection{วิเคราะห์วิธีแก้ปัญหาที่มีอยู่ในปัจจุบัน}
หลังจากการวิเคราะห์ปัญหา พวกเราได้ทำการค้นคว้าหาวิธีการแก้ปัญหาที่มีอยู่ในปัจจุบันว่ามีวิธีใดบ้างและแต่ละวิธีสามารถแก้ปัญหาได้หรือไม่ ซึ่งได้ผลลัพธ์ดังนี้
\begin{itemize}
    \item การเรียนรู้จากหนังสือเรียน - เราได้พบว่าหนังสือเรียนส่วนใหญ๋จะสอนเศษส่วนด้วยการใช้ Area Model 
    ทำให้เด็กส่วนใหญ่ติดการมองเศษส่วนในรูปของ เค้ก พิซซ่า หรือตารางสี่เหลี่ยม และไม่สามารถอ่านค่าจากรูปที่แตกต่างจากที่เคยเจอมาในหนังสือได้
    นอกจากนั้น สิ่งนี้สามารถทำให้เด็กเข้าใจเศษส่วนที่มีค่าไม่เกิน 1 ได้ แต่จะยากในการนำไปใช้กับเรื่องที่ซับซ้อนกว่านี้ เช่น การบวกลบเศษส่วนที่มีค่ามากๆ และเศษเกิน เป็นต้น
    \item การใช้สิ่งรอบตัวเป็นสื่อการสอน -  วิธีนี้จะเน้นการทำให้เด็กได้ทำกิจกรรมที่ได้ลงมือทำเองและใช้จินตนาการของตัวเองในการแก้ปัญหาเศษส่วน เช่น 
    การให้เด็กลองพักกระดาษเป็นหลายๆส่วน หรือการให้เด็กลองเปรียบเทียบความยาวของไม้บรรทัดด้วยดินสอ เป็นต้น 
    การสอนเช่นนี้จะทำให้เด็กไม่ยึดติดว่าเศษส่วนจะต้องเป็นภาพใดภาพหนึ่งและได้เข้าใจหลักการของเศษส่วน แต่ก็มีข้อจำกัดคือการเตรียมกิจกรรมนั้นมีขั้นตอนและอุปกรณ์ที่มาก จึงอาจไม่สะดวกในการนำไปใช้
    \item การใช้ Software - วิธีนี้สามารถใช้ในการเรียนรู้ได้ไม่ต่างจาก 2 วิธีก่อนหน้าด้วยการใช้แค่อุปกรณ์เช่น คอมพิวเตอร์พกพา สมาร์ตโฟน หรือแท็บเล็ต เพียงเครื่องเดียว แต่มีข้อจำกัดคือ 
    Software ที่ใช้ในการเรียนรู้เศษส่วนนั้นไม่เป็นที่แพร่หลาย และใช้ภาษาต่างประเทศเป็นหลัก ทำให้เด็กเข้าถึงได้ยาก
\end{itemize}
 
\subsection{สรุปปัญหาเบื้องต้น}
ปัญหาที่พบจากการวิเคราะห์ปัญหาและวิธีแก้ปัญหาที่มีอยู่ในปัจจุบัน คือ การเข้าใจและเห็นภาพที่ถูกต้องของเศษส่วนนั้นเป็นสิ่งที่ทำได้ยากสำหรับเด็ก
และวิธีการแก้ปัญหาบางวิธีในปัจจุบันยังไม่สามารถเข้าถึงเด็กได้ง่าย ทำให้ไม่สามารถแก้ปัญหาที่มีอยู่ได้อย่างมีประสิทธิภาพ

\section{การลงพื้นที่สำรวจ}

\subsection{เลือกพี้นที่สำรวจ}
กลุ่มของเราได้เลือกโรงเรียนที่มีการสอนนักเรียนระดับชั้นประถมศึกษาตอนปลายในเขตเชียงใหม่ทั้งสิ้น 6 \\โรงเรียน ได้แก่ โรงเรียนพิงครัตน์, โรงเรียนพุทธิโศภน, โรงเรียนดาราวิทยาลัย, โรงเรียนบ้านเชิงดอยสุเทพ, โรงเรียนโกวิทธำรงเชียงใหม่ และโรงเรียนสาธิตมหาวิทยาลัยเชียงใหม่
 ซึ่งแต่ละโรงเรียนนั้นมีระยะทางที่ไม่ไกลจากมหาวิทยาลัยเชียงใหม่มากนักและสามารถทำการติดต่อขออนุญาตจากทางโรงเรียนได้

\subsection{วางแผนสัมภาษณ์}
\begin{itemize}
    \item นักเรียน - ถามถึงวิธีคิด เพื่อทราบถึงความเข้าใจเกี่ยวกับเศษส่วน, ถามถึงความรู้สึกที่มีต่อวิชาคณิตศาสตร์ เพื่อทราบมุมมองที่เด็กมีต่อวิชาคณิตศาสตร์, 
    ถามถึงอุปกรณ์ที่ชอบใช้ เพื่อใช้ในการกำหนดอุปกรณ์หลักของโครงงานนี้
    \item ครู - ถามถึงวิธีการสอนและเทคนิคการสอน เพื่อทราบถึงวิธีที่มีประสิทธิภาพในการสอนเศษส่วนจากมุมมองของคุณครูโดยตรง, 
    ถามถึงอุปกรณ์ที่ใช้สอน เพื่อใช้ในการกำหนดอุปกรณ์หลักของโครงงานนี้
\end{itemize}

\subsection{วางแผนออกสำรวจ}
การออกไปสำรวจในแต่ละโรงเรียนนั้น ทำเพื่อสำรวจปัญหาที่เกิดขึ้นในการเรียนการสอนวิชาคณิตศาสตร์เรื่องเศษส่วน โดยเราได้วางแผนการสำรวจดังนี้
\begin{itemize}
    \item สัมภาษณ์นักเรียน 6 คน เนื่องจากเราต้องเข้าไปสัมภาษณ์นักเรียนในช่วงเวลาเรียน จึงคิดว่าการสัมภาษณ์นักเรียนจำนวนเท่านี้สามารถทำให้เสร็จได้ภายในเวลา 2 ชั่วโมง
    \item สัมภาษณ์ครู 1 คน เพราะในโรงเรียนขนาดเล็กมีครูประจำวิชาต่อชั้นเพียง 1 คน จึงเลือกสัมภาษณ์ครูจำนวนเท่านี้
\end{itemize}
