\chapter{\ifproject%
\ifenglish Project Structure and Methodology\else โครงสร้างและขั้นตอนการทำงาน\fi
\else%
\ifenglish Project Structure\else โครงสร้างของโครงงาน\fi
\fi
}

ในบทนี้จะกล่าวถึงขั้นตอนการดำเนินงานในโครงงานนี้

\makeatletter

% \renewcommand\section{\@startsection {section}{1}{\z@}%
%                                    {13.5ex \@plus -1ex \@minus -.2ex}%
%                                    {2.3ex \@plus.2ex}%
%                                    {\normalfont\large\bfseries}}

\makeatother
%\vspace{2ex}
% \titleformat{\section}{\normalfont\bfseries}{\thesection}{1em}{}
% \titlespacing*{\section}{0pt}{10ex}{0pt}

\section{การค้นคว้าข้อมูล}

\subsection{วิเคราะห์ปัญหา}
 เริ่มจากการตั้งข้อสงสัยว่า การเรียนรู้วิชาคณิตศาสตร์เรื่องเศษส่วนสำหรับเด็กนักเรียนชั้นประถมศึกษาในปัจจุบันนั้นมีปัญหาอย่างไรบ้าง
 หลังจากนั้นเราได้ติดต่อพูดคุยกับเจ้าหน้าที่ของ สสวท. เพื่อสอบถามเกี่ยวกับข้อสงสัยนี้และได้ข้อมูลว่า การเรียนรู้วิชาคณิตศาสตร์เรื่องเศษส่วนนั้นเป็นปัญหาที่พบในเด็กนักเรียนชั้นประถมศึกษาทั่วโลก
 เนื่องจากเนื้อหาในเรื่องนี้มีความเป็นนามธรรมที่สูง มีรูปแบบจำนวนที่ต่างจากคณิตศาสตร์ในเรื่องก่อนๆ และอาจมีหลักการที่ดูขัดแย้งกับสิ่งที่เขาเคยเรียนมา จึงยากที่จะทำให้เด็กทุกคนเข้าใจพร้อมๆกันและไม่สามารถทำเด็กทุกคนเข้าใจเรื่องนี้ด้วยวิธีสอนเดียวกันได้

\subsection{วิเคราะห์วิธีแก้ปัญหาที่มีอยู่ในปัจจุบัน}
เราได้ทำการค้นหาวิธีที่ใช้สอนวิชาคณิตศาสตร์เรื่องเศษส่วนในปัจจุบัน ได้แก่
\begin{enumerate}
    \item การเรียนตามหนังสือ
    \item การลงมือทำกิจกรรม
\end{enumerate}

\subsection{สรุปปัญหาเบื้องต้น}

\section{การลงพื้นที่สำรวจ}

\subsection{เลือกพี้นที่สำรวจ}

\subsection{วางแผนออกสำรวจ}


