\chapter{\ifenglish Background Knowledge and Theory\else ทฤษฎีที่เกี่ยวข้อง\fi}

\hspace*{2em}การทำโครงงาน เริ่มต้นด้วยการศึกษาค้นคว้า ทฤษฎีที่เกี่ยวข้อง หรือ งานวิจัย/โครงงาน ที่เคยมีผู้นำเสนอไว้แล้ว ซึ่งเนื้อหาในบทนี้ก็จะเกี่ยวกับการอธิบายถึงสิ่งที่เกี่ยวข้องกับโครงงาน เพื่อให้ผู้อ่านเข้าใจเนื้อหาในบทถัดๆ ไปได้ง่ายขึ้น

\section{Fraction models}
\hspace*{2em}ลักษณะของเศษส่วนที่ใช้สื่อความหมายมีหลายแบบ ซึ่งแต่ละแบบนั้นมีข้อดีข้อเสียที่แตกต่างกันไป อีกทั้งยังมีความยากง่ายในการเข้าใจที่ต่างกัน โดยแบบที่ใช้อ้างอิงมี 3 แบบ เนื่องจากเป็นรูปแบบที่ถูกนำมาใช้ในการสอนมากที่สุด ซึ่งประกอบไปด้วย

\subsection{Area model}
\hspace*{2em}Area model เป็นการใช้รูปทรงเรขาคณิตต่างๆ เช่น วงกลม สี่เหลี่ยม หรือรูปสามเหลี่ยม มาแบ่งส่วนเพื่อแสดงความหมายของเศษส่วน
โดย Area model นั้นเป็นแบบที่พบได้บ่อยที่สุดในหนังสือเรียนคณิตศาสตร์สำหรับเด็กประถมศึกษาเนื่องจากการที่สามารถสื่อความหมายของเศษส่วนได้ง่ายผ่านการแบ่งเค้กหรือพิซซ่า อีกทั้งการใช้รูปวัตถุเหล่านี้ยังช่วยให้เด็กสามารถเข้าใจแนวคิดของเศษส่วนที่เป็นส่วนหนึ่งของจำนวนเต็มได้ง่ายขึ้น
\\\hspace*{2em}อย่างไรก็ตาม Area model นั้นมีข้อจำกัดในเรื่องของการสื่อความหมายของเศษส่วนที่มีค่ามากกว่า 1 หรือเศษเกิน ซึ่งอาจทำให้เด็กเกิดความสับสนได้เมื่อต้องเรียนรู้เนื้อหาของเศษเกินนั่นเอง
\begin{figure}[h!tbp]
\begin{centering}
\includegraphics{Area_model.png}
\end{centering}
\caption[Area model]{Area model of fraction 3/4 \cite{ejmste}}
\end{figure}

\subsection{Set model}
\hspace*{2em}Set model เป็นการใช้กลุ่มของวัตถุที่เหมือนกันมาแบ่งกลุ่มเพื่อแสดงความหมายของเศษส่วน เช่น การใช้ลูกปัดสีแดงและสีขาวมาแบ่งกลุ่มเพื่อแสดงเศษส่วน
โดย Set model นั้นช่วยให้เด็กสามารถเข้าใจแนวคิดของเศษส่วนในแง่ของการแบ่งกลุ่มวัตถุที่เหมือนกันได้ดี อีกทั้งยังช่วยให้เด็กเห็นภาพรวมของเศษส่วนในรูปแบบที่แตกต่างจาก Area model จึงถูกนิยมใช้ในการสอนเศษส่วนเช่นกัน
\\\hspace*{2em}อย่างไรก็ตาม Set model นั้นทำให้เกิดความสับสนได้หากมีการคำนวนเข้ามาเกี่ยวข้อง เช่น การบวกลบเศษส่วนที่มีตัวส่วนไม่เท่ากัน เพราะ Set model นั้นไม่สามารถแสดงความสัมพันธ์ระหว่างเศษส่วนได้อย่างชัดเจนนั่นเอง
\begin{figure}[h!tbp]
\begin{center}
\includegraphics{Set_model.png}
\end{center}
\caption[Set model]{Set model of fraction 3/4 \cite{ejmste}}
\end{figure}

\subsection{Length model}
\hspace*{2em}Length model เป็นการใช้เส้นตรงเพื่อแสดงความหมายของเศษส่วน โดยจะเป็นการใช้เส้นตรงยาว 1 หน่วยมาแบ่งเป็นส่วนๆ
โดย Length model นั้นช่วยให้เด็กสามารถเข้าใจแนวคิดของเศษส่วนผ่านการวัดความยาว อีกทั้งยังช่วยให้เด็กเห็นภาพรวมของเศษส่วนในรูปแบบที่ทำให้สามารถนำไปใช้ในการคำนวนได้ง่ายขึ้น
นอกจากนี้ Length model ยังเป็น model ที่ช่วยให้เด็กเข้าใจแนวคิดของเศษส่วนได้ดีที่สุดเมื่อเทียบกับ model อื่นๆ\cite{ejmste}
\\\hspace*{2em}อย่างไรก็ตามการเรียนรู้ด้วย Length model นั้นจำเป็นที่จะต้องมีความเข้าใจพื้นฐานเกี่ยวกับการวัดความยาวและหน่วยวัดต่างๆ มาก่อน จึงอาจทำให้เด็กที่ยังไม่มีพื้นฐานเหล่านี้เกิดความสับสนได้
\begin{figure}[h!tbp]
\begin{center}
\includegraphics{Length_model.png}
\end{center}
\caption[Length model]{Length model of fraction 3/4 \cite{ejmste}}
\end{figure}


% This code demonstrates how to get a landscape table or figure. It
% uses the package lscape to turn everything but the page number into
% landscape orientation. Everything should be included within an
% \afterpage{ .... } to avoid causing a page break too early.


\section{\ifenglish%
\ifcpe CPE \else ISNE \fi knowledge used, applied, or integrated in this project
\else%
ความรู้ตามหลักสูตรซึ่งถูกนำมาใช้หรือบูรณาการในโครงงาน
\fi
}

    \subsection{Database}
    \hspace*{2em}โครงงานนี้ได้นำความรู้ด้านการออกแบบฐานข้อมูลมาใช้ในการวางโครงสร้างฐานข้อมูลของแอปพลิเคชัน เพื่อให้สามารถจัดเก็บข้อมูลผู้ใช้และข้อมูลการเรียนรู้ได้อย่างมีประสิทธิภาพ
    เนื่องจากการเรียนรู้ผ่านแอปพลิเคชันนั้น จำเป็นต้องมีการบันทึกความก้าวหน้าของผู้ใช้ และเก็บข้อมูลเนื้อหาที่ผู้ใช้ได้เรียนรู้ไปแล้ว เพื่อให้การกลับมาใช้งานในครั้งถัดไปเป็นไปอย่างราบรื่น
    \subitem Database ประเภท... (ปัจจุบันเป็น Placeholder เนื่องจากยังไม่ได้เลือกใช้ฐานข้อมูลตัวใดตัวหนึ่ง)
    
    \subsection{Human-Computer Interaction (HCI)}
    \hspace*{2em}โครงงานนี้การออกแบบและพัฒนา UX/UI มีความจำเป็นอย่างมาก เนื่องจากกลุ่มเป้าหมายของโครงงานนี้คือเด็กนักเรียนระดับประถมศึกษา
    ดังนั้นการออกแบบ UX/UI ที่เหมาะสมกับกลุ่มเป้าหมายจะทำให้เด็กๆ สามารถใช้งานแอปพลิเคชั่นได้อย่างมีประสิทธิภาพและสนุกสนาน
    โดยได้ใช้ความรู้ด้าน HCI ในการออกแบบ UX/UI ดังนี้
    \subitem User-centered design - การออกแบบที่เน้นผู้ใช้เป็นศูนย์กลาง โดยการทำความเข้าใจความต้องการและพฤติกรรมของผู้ใช้ เพื่อให้การออกแบบตอบสนองต่อความต้องการของผู้ใช้ได้อย่างเหมาะสม
    \subitem Usability principles - การนำหลักการใช้งานง่ายมาใช้ในการออกแบบ เช่น การใช้ลูกศรหรือไฮไลต์ เพื่อให้ผู้ใช้สามารถใช้งานแอปพลิเคชันได้อย่างง่ายยิ่งขึ้น

\section{\ifenglish%
Extracurricular knowledge used, applied, or integrated in this project
\else%
ทฤษฎีที่เกี่ยวข้องกับการพัฒนาระบบ (System development theory)
\fi
}

\subsection {Software Development Life Cycle (SDLC)}
    \hspace*{2em}โดย SDLC ที่เลือกใช้ในโครงงานนี้คือแบบ Agile Model ซึ่งเป็นกระบวนการพัฒนาซอฟต์แวร์แบบทำไปปรับไป โดยจะมีการปรับปรุงและพัฒนาระบบอย่างต่อเนื่องตามความต้องการของผู้ใช้ โดยในแต่ละรอบของการพัฒนาจะมีการออกแบบ การพัฒนา การทดสอบ การส่งมอบระบบ และทบทวนสิ่งที่ทำไปในรอบนั้นๆเพื่อปรับปรุงในรอบถัดไป
    \\\hspace*{2em}เนื่องจากโครงงานนี้มีความต้องการที่อาจเปลี่ยนแปลงได้ตลอดเวลา และต้องการการตอบสนองที่รวดเร็วต่อความต้องการของผู้ใช้ Agile Model จึงเป็นตัวเลือกที่เหมาะสม
    ซึ่งในขั้นตอนแต่ละขั้นตอนของ Agile Model ที่กลุ่มของเราใช้นั้นมีรายละเอียดดังนี้
    \begin{itemize}
        \item Planning - วางแผนการพัฒนาและกำหนดความต้องการของระบบ
        \item Design - ออกแบบระบบตามความต้องการที่ได้จากขั้นตอน Planning
        \item Development - พัฒนาระบบตามแบบที่ได้ออกแบบไว้
        \item Testing - ทดสอบระบบเพื่อหาข้อผิดพลาดและปรับปรุงระบบ
        \item Deployment - การให้ผู้ใช้ทดลองใช้งานระบบ
        \item Review - ทบทวนสิ่งที่ทำไปในรอบนั้นๆ เพื่อปรับปรุงในรอบถัดไป
    \end{itemize}
    \begin{figure}[h!tbp]
    \begin{centering}
    \includegraphics{Agile_Model.jpg}
    \end{centering}
    \caption[Agile model]{Visual explaination of Agile model}
    \end{figure}
