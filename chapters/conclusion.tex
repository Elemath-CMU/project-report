\chapter{\ifenglish Conclusions and Discussions\else บทสรุปและข้อเสนอแนะ\fi}

\section{\ifenglish Conclusions\else สรุปผล\fi}

ข้อจำกัดในการทำโปรเจกต์นี้ คือจำเป็นต้องทำการคำนึงถึงเนื้อหาที่เหมาะสมกับกลุ่มเป้าหมายเป็นหลักอยู่เสมอ รวมถึงการออกแบบ UX/UI ที่ผู้ใช้สามารถใช้งานได้จริง นอกจากนี้ยังมีปัจจัยด้านเวลาที่จำกัด ส่งผลให้ฟังก์ชั่นที่เลือกทำนั้นมีได้จำกัด

\section{\ifenglish Challenges\else ปัญหาที่พบและแนวทางการแก้ไข\fi}

ในการทำโครงงานนี้ พบว่าเกิดปัญหาหลักๆ ดังนี้
\begin{itemize}
    \item การออกแบบ UX/UI: ที่เหมาะสมกับกลุ่มเป้าหมาย เนื่องจากกลุ่มเป้าหมายเป็นนักเรียนระดับประถม การออกแบบต้องคำนึงถึงความง่ายในการใช้งานและความน่าสนใจ เพื่อให้เด็กๆรู้สึกอยากใช้งานแอปพลิเคชัน
    \item การเลือกเนื้อหาที่เหมาะสม: เนื้อหาที่นำเสนอในแอปพลิเคชันต้องสอดคล้องกับหลักสูตรการเรียนรู้ของเด็กนักเรียน และต้องมีความน่าสนใจเพื่อกระตุ้นความสนใจในการเรียนรู้
    \item การจัดการเวลา: เนื่องจากเวลาที่มีจำกัด จึงต้องมีการวางแผนการทำงานอย่างมีประสิทธิภาพ เพื่อให้สามารถพัฒนาแอปพลิเคชันได้ตามเป้าหมายที่ตั้งไว้
\end{itemize}

\section{\ifenglish%
Suggestions and further improvements
\else%
ข้อเสนอแนะและแนวทางการพัฒนาต่อ
\fi
}

ข้อเสนอแนะเพื่อพัฒนาโครงงานนี้ต่อไป มีดังนี้
\begin{itemize}
    \item เพิ่มขอบเขตเนื้อหา โดยขยายขอบเขตเนื้อหาที่ครอบคลุมมากขึ้น เช่น การคูณและการหารเศษส่วน เพื่อให้ผู้เรียนได้รับความรู้ที่หลากหลายมากขึ้น
    \item เพิ่มฟีเจอร์การติดตามความก้าวหน้า โดยพัฒนาระบบที่ช่วยให้ผู้เรียนสามารถติดตามความก้าวหน้าในการเรียนรู้ของตนเองได้ เช่น การบันทึกคะแนนหรือการแสดงผลการทำแบบฝึกหัด 
\end{itemize}