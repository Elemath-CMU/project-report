\chapter{\ifproject%
\ifenglish Experimentation and Results\else การทดลองและผลลัพธ์\fi
\else%
\ifenglish System Evaluation\else การประเมินระบบ\fi
\fi}
การประเมินระบบของโครงงาน เพื่อวัดความสามารถและประสิทธิภาพของสื่อการสอนที่พัฒนาขึ้นมา จะทดสอบโดยการแบ่งกลุ่มตัวอย่่างนักเรียนออกเป็น 2 กลุ่ม ดังนี้

\section{นักเรียนที่เคยสัมภาษณ์ในช่วงเก็บข้อมูลเบื้องต้น}
\begin{enumerate}
    \item ให้นักเรียนทำโจทย์ปัญหา pre-test เพื่อวัดความเข้าใจพื้นฐานเรื่องเศษส่วน
    \item ให้นักเรียนทดลองใช้ Web Application ของเรา
    \item ใช้คำถามชุดเดิมเพื่อตรวจสอบความสามารถของนักเรียนในการ
        \subitem วาดรูปแสดงเศษส่วนเป็น Length Model
        \subitem อธิบายการเปรียบเทียบเศษส่วนพร้อมเหตุผล
        \subitem อธิบายการบวกเศษส่วนที่ไม่เท่ากัน พร้อมเหตุผลว่าทำไมต้องทำตัวส่วนให้เท่าก่อน
        \subitem ระบุตำแหน่งของ ¼ บนเส้นจำนวน
    \item ให้นักเรียนทำโจทย์ปัญหา post-test เพื่อวัดผลการเรียนรู้หลังจากใช้สื่อการสอน
\end{enumerate}

\section{นักเรียนกลุ่มใหม่ที่ไม่เคยสัมภาษณ์มาก่อน}
\begin{enumerate}
    \item ให้นักเรียนทำโจทย์ปัญหา pre-test เพื่อวัดความเข้าใจพื้นฐานเรื่องเศษส่วน
    \item ให้นักเรียนทดลองใช้ Web Application ของเรา
    \item ใช้คำถามชุดเดียวกับกลุ่มแรกเพื่อตรวจสอบความสามารถของนักเรียนในการ
        \subitem วาดรูปแสดงเศษส่วนเป็น Length Model
        \subitem อธิบายการเปรียบเทียบเศษส่วนพร้อมเหตุผล
        \subitem อธิบายการบวกเศษส่วนที่ไม่เท่ากัน พร้อมเหตุผลว่าทำไมต้องทำตัวส่วนให้เท่าก่อน
        \subitem ระบุตำแหน่งของ ¼ บนเส้นจำนวน
    \item ให้นักเรียนทำโจทย์ปัญหา post-test เพื่อวัดผลการเรียนรู้หลังจากใช้สื่อการสอน
\end{enumerate}

การประเมิน Web Application ในลักษณะนี้ช่วยให้สามารถตรวจสอบได้ว่าสื่อการสอนสนับสนุนการเรียนรู้เรื่องเศษส่วนได้จริงหรือไม่ 
โดยเปรียบเทียบความเข้าใจของนักเรียนก่อนและหลังใช้สื่อ พร้อมสังเกตพฤติกรรมและปัญหาที่พบระหว่างใช้งาน ทำให้สามารถประเมินประสิทธิภาพของ Web Application ได้อย่างชัดเจน 