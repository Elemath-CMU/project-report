\chapter{\ifenglish Introduction\else บทนำ\fi}
ในปัจจุบัน การเรียนรู้เรื่อง “เศษส่วน” สำหรับนักเรียนระดับประถมศึกษายังคงเป็นเรื่องที่เข้าใจได้ยาก เนื่องจากเป็นแนวคิดเชิงนามธรรมที่นักเรียนไม่คุ้นเคย และสื่อการสอนที่มีอยู่ส่วนใหญ่มักมุ่งเน้นการผลลัพธ์ เช่น การหาคำตอบที่ถูกต้องหรือการทำโจทย์ให้ได้คะแนน มากกว่าการสร้างความเข้าใจในหลักการและกระบวนการคิด ทำให้นักเรียนไม่สามารถเชื่อมโยงเศษส่วนกับสถานการณ์ในชีวิตจริงได้

ด้วยเหตุนี้ ผู้จัดทำจึงเสนอแนวคิดในการพัฒนาสื่อการสอนแบบเปิด เพื่อให้นักเรียนได้ลงมือทดลอง คิดแก้ปัญหา และค้นพบความรู้ด้วยตนเองผ่านการจำลองสถานการณ์ต่างๆ ที่เกี่ยวข้องกับเศษส่วน โดยเน้นให้เด็กได้สังเกตและเข้าใจแนวคิดหลักโดยไม่ถูกจำกัดด้วยวิธีการคงที่หรือตัวเลือกที่ตายตัว

\section{\ifenglish Project rationale\else ที่มาของโครงงาน\fi}
\begin{enumerate}
    \item นักเรียนระดับประถมศึกษายังมีความเข้าใจในเรื่องเศษส่วนไม่มากนัก โดยนักเรียนสามารถบอกได้เพียงว่าเศษส่วนคืออะไร แต่ไม่สามารถอธิบายความหมายออกมาได้
    \item สื่อการสอนที่มีอยู่ยังไม่ตอบโจทย์การเรียนรู้ของนักเรียน
\end{enumerate}

\section{\ifenglish Objectives\else วัตถุประสงค์ของโครงงาน\fi}
\begin{enumerate}
    \item สร้าง Web Application เชิงโต้ตอบ (Interactive) สำหรับการเรียนรู้ในเรื่องของเศษส่วน เกี่ยวกับความเข้าใจเบื้องต้น การบวก และการลบ โดยใช้ Length model
    \item สร้างเครื่องมือการเรียนรู้ ที่ให้ผู้เรียนสามารถทดลอง แบ่ง แทนค่า และจัดการเศษส่วนด้วยภาพ
\end{enumerate}

\section{\ifenglish Project scope\else ขอบเขตของโครงงาน\fi}

\subsection{\ifenglish Hardware scope\else ขอบเขตด้านฮาร์ดแวร์\fi}

\subsection{\ifenglish Software scope\else ขอบเขตด้านซอฟต์แวร์\fi}

\section{\ifenglish Expected outcomes\else ประโยชน์ที่ได้รับ\fi}

\section{\ifenglish Technology and tools\else เทคโนโลยีและเครื่องมือที่ใช้\fi}

\subsection{\ifenglish Hardware technology\else เทคโนโลยีด้านฮาร์ดแวร์\fi}

\subsection{\ifenglish Software technology\else เทคโนโลยีด้านซอฟต์แวร์\fi}

\section{\ifenglish Project plan\else แผนการดำเนินงาน\fi}

\begin{plan}{6}{2024}{2}{2025}
    \planitem{7}{2024}{8}{2024}{User interview plan}
    \planitem{8}{2024}{1}{2025}{User interview}
    \planitem{2}{2025}{2}{2025}{Project design setup}
    \planitem{12}{2024}{1}{2025}{ทดสอบ}
\end{plan}

\section{\ifenglish Roles and responsibilities\else บทบาทและความรับผิดชอบ\fi}
อธิบายว่าในการทำงาน นศ. มีการกำหนดบทบาทและแบ่งหน้าที่งานอย่างไรในการทำงาน จำเป็นต้องใช้ความรู้ใดในการทำงานบ้าง

\section{\ifenglish%
Impacts of this project on society, health, safety, legal, and cultural issues
\else%
ผลกระทบด้านสังคม สุขภาพ ความปลอดภัย กฎหมาย และวัฒนธรรม
\fi}

แนวทางและโยชน์ในการประยุกต์ใช้งานโครงงานกับงานในด้านอื่นๆ รวมถึงผลกระทบในด้านสังคมและสิ่งแวดล้อมจากการใช้ความรู้ทางวิศวกรรมที่ได้
