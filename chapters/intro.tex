\chapter{\ifenglish Introduction\else บทนำ\fi}

\section{\ifenglish Project rationale\else ที่มาของโครงงาน\fi}
ในปัจจุบัน การเรียนรู้คณิตศาสตร์เป็นสิ่งสำคัญที่ช่วยพัฒนาทักษะการคิดวิเคราะห์ของผู้เรียน เนื่องจากคณิตศาสตร์เป็นพื้นฐานของการใช้เหตุผล การแก้ปัญหา 
และการเชื่อมโยงความรู้ไปสู่ศาสตร์อื่น ๆ อย่างไรก็ตาม จากผลการประเมินสมรรถนะนักเรียนระดับนานาชาติ (PISA 2022) พบว่า 
คะแนนเฉลี่ยของนักเรียนไทยต่ำกว่าค่าเฉลี่ยของประเทศสมาชิก OECD ทุกวิชา โดยเฉพาะในวิชาคณิตศาสตร์และการอ่าน ซึ่งมีคะแนนไม่ถึง 400 คะแนน 
สะท้อนให้เห็นถึงปัญหาการทำความเข้าใจแนวคิดพื้นฐานและการคิดเชิงวิเคราะห์ของผู้เรียนไทยในปัจจุบัน

เมื่อศึกษาข้อมูลเพิ่มเติมและพูดคุยกับนักวิชาการจากสถาบันส่งเสริมการสอนวิทยาศาสตร์และเทคโนโลยี (สสวท.) พบว่า หนึ่งในหัวข้อที่นักเรียนทั่วโลก 
รวมทั้งประเทศไทยมักประสบปัญหาคือ เศษส่วน แม้จะเป็นหัวข้อพื้นฐาน แต่แนวคิดของเศษส่วนเป็นนามธรรมและเข้าใจได้ยาก เด็กจำนวนมากสามารถคำนวณได้ 
แต่ไม่สามารถอธิบายได้ว่าทำไมจึงคิดเช่นนั้น ซึ่งส่งผลให้ไม่เข้าใจแก่นของคณิตศาสตร์อย่างแท้จริง

เพื่อยืนยันปัญหานี้ พวกเราได้ลงพื้นที่เก็บข้อมูลจากโรงเรียนในเขตพื้นที่ใกล้มหาวิทยาลัยเชียงใหม่ จำนวน 6 โรงเรียน โดยเลือกโรงเรียนที่มีการสอนระดับประถมศึกษาตอนปลาย 
ซึ่งเป็นช่วงวัยที่เริ่มเรียนรู้เรื่องเศษส่วน การเก็บข้อมูลประกอบด้วยการสัมภาษณ์นักเรียนชั้นประถมศึกษาปีที่ 5 จำนวน 6 คน และครูผู้สอนวิชาคณิตศาสตร์ 1 คนต่อโรงเรียน 
เพื่อให้ได้มุมมองทั้งจากผู้เรียนและผู้สอน

โดยเรามีข้อสันนิษฐานในการสัมภาษณ์นักเรียนเบื้องต้นว่า
\begin{enumerate}
    \item นักเรียนสามารถวาดรูปแสดงเศษส่วนเป็นรูปแบบ Area Model ได้
    \item นักเรียนสามารถวาดรูปแสดงเศษส่วนเป็นรูปแบบอื่นนอกจาก Area Model ได้
    \item นักเรียนสามารถอธิบายการเปรียบเทียบเศษส่วนได้
    \item นักเรียนสามารถอธิบายถึงการบวกเศษส่วนได้
    \item นันักเรียนสามารถระบุตำแหน่งของ ¼ บนเส้นจำนวนได้อย่างถูกต้อง
\end{enumerate}

จากการสัมภาษณ์นักเรียน พบว่า 
\begin{enumerate}
    \item นักเรียนส่วนใหญ่สามารถวาดรูปแสดงเศษส่วนในรูปแบบ Area Model ได้
    \item มีนักเรียนเพียงส่วนน้อยที่สามารถวาดรูปแสดงเศษส่วนในรูปแบบอื่น เช่น Length Model หรือ Set Model ได้
    \item นักเรียนส่วนใหญ่สามารถเปรียบเทียบเศษส่วนได้ แต่ไม่สามารถอธิบายเหตุผลเบื้องหลังการคิดได้อย่างชัดเจน
    \item นักเรียนส่วนใหญ่สามารถบวกเศษส่วนได้ แต่ไม่สามารถอธิบายเหตุผลเบื้องหลังการคิดได้อย่างชัดเจน
    \item นักเรียนส่วนใหญ่ไม่สามารถระบุตำแหน่งของ ¼ บนเส้นจำนวนได้อย่างถูกต้อง
\end{enumerate}

จากการสัมภาษณ์ครูผู้สอน พบว่า ครูส่วนใหญ่เริ่มต้นสอนเศษส่วนด้วยการยกตัวอย่างจากสิ่งใกล้ตัว เช่น การแบ่งพิซซ่าหรือเค้ก เพื่อให้เด็กมองเห็นภาพได้ง่ายขึ้น 
เนื่องจากก่อนหน้านั้นนักเรียนจะคุ้นเคยกับ “จำนวนเต็ม” มากกว่า การมองเห็นปริมาณที่ไม่เต็มหน่วยจึงเป็นสิ่งที่ท้าทายสำหรับเด็กในช่วงแรก 
และหนังสือเรียนของ สสวท. ก็ยังคงใช้การอธิบายด้วย Area Model ซึ่งเป็นการแบ่งพื้นที่เป็นส่วน ๆ 
ทำให้เด็กบางคนไม่สามารถเชื่อมโยงแนวคิดนี้ไปสู่การใช้เศษส่วนในบริบทอื่น ๆ ได้

สิ่งที่คุณครูหลายคนกล่าวตรงกันคือ “ความยากที่สุดของการสอนเศษส่วนคือการทำให้เด็กเห็นภาพว่าเศษส่วนคืออะไร” โดยเฉพาะในเรื่องของการคูณและหาร 
ซึ่งคุณครูมักรู้สึกว่ายากต่อการอธิบายให้เด็กเข้าใจ ขณะที่สำหรับนักเรียนเอง กลับพบว่าการบวกและลบเศษส่วนเป็นหัวข้อที่เข้าใจยากที่สุด และในบางโรงเรียน 
ครูผู้สอนมีการปรับลำดับเนื้อหาการสอนใหม่ เพื่อให้สอดคล้องกับลำดับความเข้าใจของผู้เรียนจริง เนื่องจากหลักสูตรของ สสวท. มีลักษณะการจัดเรียงเนื้อหาที่ไม่ต่อเนื่อง 
ทำให้เด็กบางคนยังไม่พร้อมจะเรียนรู้หัวข้อที่ซับซ้อนขึ้น

ข้อมูลที่ได้จากครูและนักเรียนสะท้อนภาพเดียวกันว่า การสอนเศษส่วนยังขาดสื่อหรือเครื่องมือที่ช่วยให้เด็กเข้าใจเศษส่วนอย่างเป็นรูปธรรม 
ปัญหานี้จึงกลายเป็นจุดเริ่มต้นให้พวกเราคิดต่อยอดไปสู่การพัฒนาสื่อการสอนรูปแบบใหม่

นอกจากนี้ จากการสัมภาษณ์นักเรียนและครูผู้สอน พบว่านักเรียนส่วนใหญ่มีสมาร์ตโฟนเป็นของตนเอง และมักใช้เวลาเล่นเกมหรือสื่อสังคมออนไลน์ เช่น 
TikTok, Roblox, PUBG และ Free Fire เป็นประจำ ซึ่งสะท้อนให้เห็นว่าเทคโนโลยีเป็นส่วนหนึ่งในชีวิตประจำวันของนักเรียน 
พวกเราจึงเห็นว่า นี่เป็นโอกาสที่ดีในการเปลี่ยนเวลาที่เด็ก ๆ ใช้มือถือให้กลายเป็นช่วงเวลาการเรียนรู้นอกห้องเรียน ผ่านการทำสื่อการสอนที่อยู่ในรูปแบบของเทคโนโลยีดิจิทัล

ดังนั้น พวกเราจึงตั้งเป้าหมายที่จะพัฒนา สื่อการสอนในรูปแบบ Web Application ที่สามารถใช้งานได้ทั้งบนสมาร์ตโฟนและแท็บเล็ต เพื่อให้นักเรียนสามารถเรียนรู้ได้ทุกที่ทุกเวลา 
สื่อการสอนนี้จะเน้นการอธิบายแนวคิดเศษส่วนผ่าน Length Model โดยอ้างอิงจากงานวิจัย The effects of using length models to teach fraction and decimal translation 
ซึ่งพบว่าการเรียนรู้ด้วย Length Model ช่วยให้ผู้เรียนเข้าใจความหมายของเศษส่วนได้ชัดเจนกว่าแบบ Area Model 
และสามารถเชื่อมโยงแนวคิดการใช้เศษส่วนไปในบริบทอื่น ๆ ได้ง่ายกว่าแบบ Area Model

\section{\ifenglish Objectives\else วัตถุประสงค์ของโครงงาน\fi}
จากปัญหาที่ได้กล่าวมาในหัวข้อ 1.1 เราจึงมีความตั้งใจที่จะจัดทําโครงงานนี้ขึ้นมาเพื่อแก้ไขปัญหาดังกล่าว โดยมีวัตถุประสงค์และขอบเขตของโครงงานดังนี้
\begin{enumerate}
    \item พัฒนาสื่อการสอนในรูปแบบ Web Application ที่สามารถใช้งานได้ทั้งบนสมาร์ตโฟนและแท็บเล็ต โดยใช้ Length Model ในการอธิบายแนวคิดเศษส่วน
    \item ทำให้นักเรียนสามารถวาดรูปแสดงเศษส่วนในรูปแบบ Length Model ได้
    \item ทำให้นักเรียนสามารถอธิบายการเปรียบเทียบเศษส่วนได้อย่างชัดเจน
    \item ทำให้นักเรียนสามารถอธิบายถึงการบวกเศษส่วนได้อย่างชัดเจน
    \item ทำให้นักเรียนสามารถเข้าใจเศษส่วนในระบบเส้นจำนวนได้
\end{enumerate}

\section{\ifenglish Project scope\else ขอบเขตของโครงงาน\fi}

\subsection{\ifenglish Hardware scope\else ขอบเขตด้านฮาร์ดแวร์\fi}
\begin{enumerate}
    \item นักเรียนสามารถใช้งานสื่อการสอนผ่านอุปกรณ์สมาร์ตโฟน และแท็บเล็ต ที่มีการเชื่อมต่ออินเทอร์เน็ต
    \item ครูผู้สอนสามารถใช้งานสื่อการสอนผ่านอุปกรณ์สมาร์ตโฟน และแท็บเล็ต ที่มีการเชื่อมต่ออินเทอร์เน็ต
\end{enumerate}

\subsection{\ifenglish Software scope\else ขอบเขตด้านซอฟต์แวร์\fi}
\begin{enumerate}
    \item นักเรียนสามารถใช้งานสื่อการสอนผ่านเว็บเบราว์เซอร์ เช่น Google Chrome, Safari, Microsoft Edge
    \item ครูผู้สอนสามารถใช้งานสื่อการสอนผ่านเว็บเบราว์เซอร์ เช่น Google Chrome, Safari, Microsoft Edge
\end{enumerate}

\section{\ifenglish Expected outcomes\else ประโยชน์ที่ได้รับ\fi}
\begin{enumerate}
    \item มีสื่อการสอนที่ช่วยเสริมการเรียนรู้เรื่องเศษส่วน ในรูปแบบ Web Application ที่สามารถใช้งานได้ทั้งบนสมาร์ตโฟนและแท็บเล็ต
    \item นักเรียนสามารถเข้าใจแนวคิดเศษส่วนผ่าน Length Model ได้
    \item นักเรียนสามารถเชื่อมโยงแนวคิดเศษส่วนไปในบริบทอื่น ๆ ได้
\end{enumerate}

\section{\ifenglish Technology and tools\else เทคโนโลยีและเครื่องมือที่ใช้\fi}

\subsection{\ifenglish Hardware technology\else เทคโนโลยีด้านฮาร์ดแวร์\fi}
\begin{itemize}
    \item คอมพิวเตอร์พกพา ระบบปฏิบัติการ Windows และ macOS
    \item สมาร์ตโฟน ระบบปฏิบัติการ Android และ iOS
    \item แท็บเล็ต ระบบปฏิบัติการ Android และ iOS
\end{itemize}

\subsection{\ifenglish Software technology\else เทคโนโลยีด้านซอฟต์แวร์\fi}
\begin{itemize}
    \item Figma: ใช้สำหรับออกแบบ UI ของสื่อการสอน
    \item Github: ใช้สำหรับ version control 
    \item Visual Studio Code: ใช้สำหรับเขียนโค้ด
    \item Web Browser: ใช้สำหรับทดสอบการทำงาน และการแสดงผลของสื่อการสอน
\end{itemize}

\section{\ifenglish Project plan\else แผนการดำเนินงาน\fi}

\begin{plan}{6}{2025}{12}{2025}
    \planitem{6}{2025}{8}{2025}{Project research}
    \planitem{9}{2025}{9}{2025}{Requirements elicitation}
    \planitem{10}{2025}{10}{2025}{Design}
\end{plan}

\section{\ifenglish Roles and responsibilities\else บทบาทและความรับผิดชอบ\fi}
\begin{itemize}
    \item นายธนภัทร เชยชมศรี มีบทบาทเป็น developer และรับผิดชอบในส่วน front-end development
    \item นายธีรภัทร์ ลําตาล มีบทบาทเป็น developer และรับผิดชอบในส่วน front-end development
    \item นางสาวพนิดา สุทธภักติ มีบทบาทเป็น UX/UI designer และรับผิดชอบในส่วน UX/UI design
    \item นายอนรรฆ สันตินรนนท์ มีบทบาทเป็น developer และรับผิดชอบในส่วน front-end development, back-end development
\end{itemize}

\section{\ifenglish%
Impacts of this project on society, health, safety, legal, and cultural issues
\else%
ผลกระทบด้านสังคม สุขภาพ ความปลอดภัย กฎหมาย และวัฒนธรรม
\fi}
พวกเรามองว่า โครงงานสื่อการสอนเศษส่วนในรูปแบบ Web Application จะช่วยให้เด็กทุกคนเข้าถึงการเรียนรู้ได้อย่างเท่าเทียม 
ไม่จำกัดเพียงโรงเรียนหรือพื้นที่ใดพื้นที่หนึ่ง ทำให้ลดความเหลื่อมล้ำทางการศึกษาในระดับประถมศึกษาได้ ถือเป็นผลกระทบด้านสังคมรูปแบบหนึ่ง